% The base formatting that I use for all math homework, tutorials, notes, etc...
\documentclass[hidelinks, 12pt]{article}    % hide the links, 12pt font

% Packages
\usepackage[margin=1in]{geometry} % set margins
\usepackage{amsmath, amssymb, amsthm}   % math formatting and symbols
\usepackage{graphicx}           % insert graphics
\usepackage{derivative}         % easy derivatives
\usepackage{fancyhdr}           % header formatting
\usepackage{hyperref}           % allow table of contents
\usepackage{physics}            % allow usage of many physics symbols
\usepackage{tikz}               % allow usage of the tikz drawing abilities

% Standard title + author
\title{CSC301 Team Contract}

\renewcommand*\contentsname{CSC301 Team Contract} % rename the table of contents
% \setcounter{secnumdepth}{0} % remove numbering from sections
% \setlength{\parindent}{0pt} % remove paragraph indentation
\setlength{\parskip}{1em}

% begin header formatting
\pagestyle{fancy}
\fancyhf{}
\fancyhead[L]{\nouppercase{\leftmark}}
\fancyhead[R]{\thepage}
\rfoot{}
% end header formatting


%----------------------------------------------------------------------------------------
%	PARAGRAPH SPACING SPECIFICATIONS
%----------------------------------------------------------------------------------------

\setlength{\parindent}{0mm} % Don't indent paragraphs

\setlength{\parskip}{2.5mm} % Whitespace between paragraphs

%----------------------------------------------------------------------------------------
%	PAGE LAYOUT SPECIFICATIONS
%----------------------------------------------------------------------------------------

\usepackage{geometry} % Required to modify the page layout

\setlength{\textwidth}{16cm} % Width of the text on the page
\setlength{\textheight}{24.5cm} % Height of the text on the page

\setlength{\oddsidemargin}{0cm} % Width of the margin - negative to move text left, positive to move it right

% Uncomment for offset margins if the 'twoside' document class option is used
%\setlength{\evensidemargin}{-0.75cm} 
%\setlength{\oddsidemargin}{0.75cm}

\setlength{\topmargin}{-1.25cm} % Reduce the top margin

%----------------------------------------------------------------------------------------
%	FONT SPECIFICATIONS
%----------------------------------------------------------------------------------------

% If you are running Apple OS X, uncomment the next 4 lines and comment/delete the block below, you will now need to compile with XeLaTeX but your document will look much better

%\usepackage[cm-default]{fontspec}
%\usepackage{xunicode}

%\setsansfont[Mapping=tex-text,Scale=1.1]{Gill Sans}
%\setmainfont[Mapping=tex-text,Scale=1.0]{Hoefler Text}

%-------------------------------------------

\usepackage[utf8]{inputenc} % Required for including letters with accents
\usepackage[T1]{fontenc} % Use 8-bit encoding that has 256 glyphs

\usepackage{avant} % Use the Avantgarde font for headings
\usepackage{mathptmx} % Use the Adobe Times Roman as the default text font together with math symbols from the Sym­bol, Chancery and Com­puter Modern fonts

%----------------------------------------------------------------------------------------
%	SECTION TITLE SPECIFICATIONS
%----------------------------------------------------------------------------------------

\usepackage{titlesec} % Required for modifying section titles

\titleformat{\section} % Customize the \section{} section title
{\sffamily\large\bfseries} % Title font customizations
{\thesection} % Section number
{16pt} % Whitespace between the number and title
{\large} % Title font size
\titlespacing*{\section}{0mm}{7mm}{0mm} % Left, top and bottom spacing around the title

\titleformat{\subsection} % Customize the \subsection{} section title
{\sffamily\normalsize\bfseries} % Title font customizations
{\thesubsection} % Subsection number
{16pt} % Whitespace between the number and title
{\normalsize} % Title font size
\titlespacing*{\subsection}{0mm}{5mm}{0mm} % Left, top and bottom spacing around the title


% Begin body of the document
\begin{document}
    \maketitle
    \pagebreak
    \tableofcontents
    \pagebreak

    \section{Methods of Communication}
        \subsection{How Members Will Communicate}
            The methods of communication will be primarily through discord. Other forms of communication can include
            (but are not limited to) call, txt, or email.

    \section{Communication Response Times}
        \subsection{Expected Response Time}
            The response times for group members is expected to be within 24 hours. While in actuality many response
            times may be shorter, group members are expected to have some sort of response within 24 hours, just so
            that other members do not end up being left in the dark to their whereabouts.

    \section{Meeting Attendance}
        \subsection{Expected Attendance}
            There will be meetings called for on discord and potentially tutorials that group members are
            expected to attend. While it may not be possible for a member to attend every meeting,
            it is expected that they follow communication response times to inform the other members
            (ideally through discord) of their absence.

    \section{Running Meetings}
        \subsection{How Meetings Are Run}
            The meetings will be primarily online through discord. While there may be some in-person meetings,
            group members are expected to either provide a reason why they cannot attend, or join that
            in-person meeting through discord virtually.

    \section{Meeting Preparation}
        \subsection{What Members Need}
            Most meetings will not need any further preparation, aside from just attendance.
            Should meetings have deliverables, it is expected that members have them ready.
            
    \pagebreak

    \section{Version Control}
        \subsection{Expectation of How Members Use Git}
            Git commit messages should be in the imperative form, following guidelines taught in CSC290.
            The guidelines can also be found at this website. \href{https://cbea.ms/git-commit/}{Click Here}.

            In addition, members will follow proper `git flow.'  \\
            The `master branch' will be used as the `production branch' for every version release. \\
            The `develop branch' will be used as the branch members merge their own feature branches to. \\
            The `feature branch' is a separate branch that developers will use to implement their features. \\
            
            The merge order will be as follows: \\
            feature $\rightarrow$ develop. \\
            develop $\rightarrow$ production, which is also the master branch
        

    \section{Division of Work}
        \subsection{Allocation of Points for Tasks}
            Each `task' will be assigned a point value depending on expected time, and reasonability. \\
            By the end of each sprint, it is expected that each group member have similar `scores' within reason.

            The `within reason' is decided by the entire group, giving members peace of mind should something come
            up that requires their immediate attention. (examples: Tests, Assignments, etc...)

        \subsection{Pair Programming}
            Group members can choose to collaborate in real time on code. The allocation of points can be
            negotiated with all collaborators in a way that is `fair.' 

    \section{Submitting Assignments}
        \subsection{The Expected Format}
            The exact format is not stated, but each pull request must give the following information in some way. \\

            \begin{itemize}
                \item What the pull request does
                \item Why the pull request should be accepted
                \item If there were new tests written for this pull request
                \item Whether of not this pull request increase the product version
            \end{itemize}
            
    \pagebreak

    \section{Contingency Planning}
        \subsection{Group member leaves group}
            In the event that a group member leaves the group, they will inform members of their plan
            to leave the group, and the remaining group members will contact the instructor for further
            information
            
        \subsection{Group Member is Unresponsive}
            If there is consistent absence with no attempt at contacting the group over a long
            duration of time, the other group members will inform the instructor/TA and wait for further
            instructions.
        
        \subsection{Plagiarism}
            It is expected that group members work with each other, and not take code from other
            students in other groups. That being said, in real life, it is not unheard of for dev teams
            to interact and share ideas and ideas with each other.

            Should code be taken from outside sources (outside from their group member), it is expected that
            the code is credited and sourced in the comments. An example is getting a solution
            from stack overflow, a textbook, or a more knowledgeable friend.
            
            In the event that it is expected that a member is plagiarizing, or cheating in some other way,
            the instructor/TA will be informed for further instructions if attempts to quell the issue
            are not successful.
            
    
    Signed: \\
    Michael Kwan \\
    Juan-Pablo Moreno \\
    Abtin Ghajarieh Sepanlou \\
    Tommy Zhang \\
    Peter Albu \\
    Xinlei Xu \\
    Aaron Huang \\
\end{document}
